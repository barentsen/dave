

\documentclass[iop,revtex4,twocolappendix]{emulateapj}
%\documentclass[preprint]{aastex}

\usepackage{amssymb}
\usepackage[usenames,dvipsnames,svgnames]{xcolor}
\usepackage{amsmath, mathrsfs}
\usepackage{amsfonts}

\definecolor{darkgreen}{rgb}{0,.5,0}
\newcommand{\chk}{ \textcolor{darkgreen}{{\tt (Check this)}}}
\newcommand{\add}{} %Indicate new text in a draft
\newcommand{\cut}{\em} %Indicate excised text in a draft

\newfont{\nf}{cmfib8 at 10pt}
\newcommand{\note}[1]{ \textcolor{blue}{ {\nf #1}}}

%Shortcuts for common symbols
\newcommand{\mj}{\,$\mathrm{M_J}$ }
\newcommand{\rj}{\,$\mathrm{R_J}$ }

\newcommand{\au}{\,AU}
\newcommand{\um}{\,$\mu$m}
\newcommand{\me}{\,M$_{\earth}$}
\newcommand{\msolar}{\,M$_{\odot}$}
\newcommand{\pdot}{$\dot{P}$}
\newcommand{\teff}{$T_{\mathrm{eff}}$}
\newcommand{\logg}{$\log{g}$}
\newcommand{\teq}{$T_{\mathrm{eq}}$}
\newcommand{\Rearth}{$R_{\oplus}$}

\newcommand{\kepler}{{\it Kepler}}

\newcommand{\Plwr}{\ensuremath{P_{\mathrm{lwr}}}}
\newcommand{\Pupr}{\ensuremath{P_{\mathrm{upr}}}}


\shortauthors{Mullally et al.}


\begin{document}

\title{Optimal Choices for the Box Least Squares Algorithm}
\shorttitle{\kepler\ False Alarms}

\author{F.~Mullally\altaffilmark{1},
}

\altaffiltext{1}{SETI/NASA Ames Research Center, Moffett Field, CA 94035, USA}

\email{fergal.mullally@nasa.gov}


%``All of science is either physics or stamp collecting''\\
%- E. Rutherford

\begin{abstract}
\end{abstract}
\keywords{}

\setlength{\parskip}{1.0ex plus0.5ex minus0.2ex}

%\section*{Version}
%\begin{verbatim}
%$Id: ms.tex 194 2016-03-10 00:13:09Z fergalm $
%\end{verbatim}

\section{Introduction}

\section{Results}

Consider a transit with period $P$ and duration $\tau$ on a star observed continuously
for a timespan $T$. We will assume, for simplicity, that we make $N$ equally spaced observations across $T$.

\subsection{Trial Period Spacing}
To perform a BLS search across a range of periods \Plwr to \Pupr, the 
$i^{\mathrm{th}}$ trial period to search is given by

\begin{equation}
P_i = \Plwr(1+ \tau/T)^i
\end{equation}

{\bf Justification}
Initially, we do not know the transit period, and we wish to select the smallest number of trial periods, $P_i$ such that at least one such $P_i$ is close enough to the true period to ensure detection of the transit signal. We define ``close enough'' to mean that the difference in phase for any two points when folded at the two periods is less than a transit duration. Let the phase of the 0th transit be $\phi_0$. If we assume we observe $n = T/P$ transits, the worst phase error will be
$t0 + n(P'-P)$. We therfore require

$$
n(P'-P) \leq \tau
$$

But $n = T/P$, so our requirement that all transits overlap implies
$$
P' < P + \leq \tau P/T
$$

Starting with \Plwr, our next trial period will be
\noindent
$P_1 = \Plwr\ + \tau \Plwr/T = \Plwr \times (1+ \tau/T)$\\
$P_2 = P_1\times (1+ \tau/T) = \Plwr \times (1+ \tau/T)^2$\\
...\\
$P_{i} = \Plwr \times (1+\tau/T)^i$\\


Choosing this period spacing will ensure that all transit mid-points lie within one transit duration for at least one trial period. In practice, this 







Fold the lightcurve at a trial period $P'$. 

\end{document}


